\begin{foreignabstract}
En una red las entidades relevantes son nodos y conexiones entre nodos, y en general el principal objetivo buscado es lograr una comunicaci\'on segura entre nodos de esta red, ya sea para redes telef\'onicas y de comunicaci\'on de datos, de transporte, arquitectura de computadores, redes de energ\'ia el\'ectrica o sistemas de comando y control. 
La optimizaci\'on relativa al costo de una red y la confiabilidad de la misma, relacionada con la supervivencia de esta, son los criterios predominantes en la selecci\'on de una soluci\'on para la mayor parte de los contextos. Un tema interesante que ha atra\'ido un gran esfuerzo es c\'omo dise\~nar topolog\'ias de red, con un uso m\'inimo de recursos de red en t\'erminos de costo que brinde una garant\'ia de confiabilidad. A pesar que por a\~nos el costo ha sido el factor primario, la confiabilidad ha ganado r\'apidamente en relevancia. Con sistemas de transmisi\'on de fibra \'optica de alta capacidad formando la columna vertebral de la mayor\'ia de las redes actuales y junto con el r\'apido desarrollo de la tecnolog\'ia de comunicaci\'on de redes y el crecimiento explosivo de las aplicaciones de Internet, la confiabilidad de la red parece cada vez m\'as importante, tanto para \'areas tradicionales como la industria de defensa, finanzas y energ\'ia, y \'areas emergentes como la computaci\'on confiable, la computaci\'on en la nube, internet de las cosas (IoT) y la pr\'oxima generaci\'on de Internet, la supervivencia del tr\'afico por sobre los fallos de red se ha convertido a\'un en m\'as cr\'itica.
En ese sentido podemos diferenciar, a grandes rasgos, dos de los principales problemas a resolver en el an\'alisis y dise\~no de topolog\'ias de red. Primeramente la obtenci\'on de una red \'optima en alg\'un sentido, siendo este definido por ejemplo mediante la obtenci\'on de la m\'axima cantidad posible de caminos disjuntos entre pares de nodos, esto sujeto a determinadas restricciones definidas seg\'un el contexto. El segundo problema es la evaluaci\'on de la confiabilidad de la red en funci\'on de las confiabilidades elementales de los nodos y conexiones entre nodos que componen la red. Estas confiabilidades elementales son probabilidades de operaci\'on asociadas a los nodos y conexiones entre nodos. Ambos problemas est\'an fuertemente relacionados, pudiendo tener que comparar en el proceso de b\'usqueda de redes \'optimas la confiabilidad entre soluciones candidatas, o luego de obtener una soluci\'on candidata tener que evaluar la confiabilidad de la misma y de esta forma descartarla o no.
El presente trabajo se centra en la resoluci\'on del problema enfocado en ambos puntos planteados. Para ello modelamos el problema de dise\~no de la topolog\'ia de red sobre la base de un modelo definido como Generalized Steiner Problem with Node-Connectivity Constraints and Hostile Reliability (GSP-NCHR) extensi\'on del m\'as conocido Generalized Steiner Problem (GSP).
El presente problema es $\mathcal{NP}$-duro, dedicamos un cap\'itulo para presentar resultados te\'oricos que lo demuestran. Nuestro objetivo es atacar de forma aproximada el modelo GSP-NCHR de tal modo de poder resolver la optimizaci\'on de la red y luego medir la confiabilidad de la soluci\'on obtenida. Para ello optamos por desarrollar la metaheur\'istica Variable Neighborhood Search (VNS). VNS es un m\'etodo potente que combina el uso de b\'usquedas locales basadas en distintas definiciones de vecindad, el cual ha sido utilizado para obtener soluciones de buena calidad en distintos problemas de optimizaci\'on combinatoria. 
En lo referente al c\'alculo de confiabilidad de la red, nuestro modelo GSP-NCHR pertenece a la clase $\mathcal{NP}$-duro, por eso desarrollamos Recursive Variance Reduction (RVR) como m\'etodo de simulaci\'on, ya que la evaluaci\'on exacta de esta medida para redes de tama\~no considerable es impracticable. 
Las pruebas experimentales fueron realizadas utilizando un conjunto amplio de casos de prueba adaptados de la librer\'ia travel salesman problem (TSPLIB), de heterog\'eneas topolog\'ias con diferentes caracter\'isticas, incluyendo instancias de hasta 400 nodos. Los resultados obtenidos indican tiempos de 
c\'omputo altamente aceptables acompa\~nados de \'optimos locales de buena calidad.

\end{foreignabstract}

Como resultado de esta tesis, se ha logrado la siguiente publicaci\'on: "A GRASP/VND Heuristic for the Generalized Steiner Problem with Node-Connectivity Constraints and Hostile Reliability" que ser\'a publicada en \textit{"Proceedings of the 8th International Conference on Variable Neighborhood Search (ICVNS Marzo 2021). Khalifa University, Abu Dhabi, U.A.E." Este art\'iculo ser\'a publicado por Springer en "Lecture Notes in Computer Science (LNCS) series."}

